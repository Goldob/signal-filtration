\documentclass{article}

\usepackage[utf8]{inputenc}
\usepackage{polski}
\usepackage[polish]{babel}
\usepackage{hyperref}
\usepackage[export]{adjustbox}
\usepackage{tabu}

\title{Zasady programowania strukturalnego II -~projektowanie}
\author{Adam Gotlib}

\begin{document}
	\textbf{Zasady programowania strukturalnego II – projektowanie} \\

\begin{tabu} spread \textwidth { | X[-1] | X | X[-1] | X[-1] | }
	\hline
 	Imię i nazwisko & Adam Gotlib & Grupa & AR-121 \\
 	\hline
 	Prowadzący & \multicolumn{3}{l |}{Damian Suski} \\
 	\hline
 	Temat projektu & \multicolumn{3}{l |}{Filtracja} \\
 	\hline
\end{tabu}

	\section{Opis projektu}
Celem projektu jest implementacja algorytmu filtrowania sygnału w dziedzinie częstotliwości.

Program występował będzie w dwóch wersjach: tekstowej oraz graficznej. W pierwszej z nich będzie on
miał minimalistyczną formę, zgodną z zasadami Filozofii Unixa\footnote{\url{https://en.wikipedia.org/wiki/Unix_philosophy}}. Nie przewiduje się interakcji z użytkownikiem - potrzebne do działania
parametry zostaną mu przekazane	przy uruchomieniu poprzez argumenty wiersza poleceń, a od razu po
wykonaniu zadania zakończy on swoje działanie.

Wersja graficzna natomiast będzie reagowała na czynności użytkownika oraz umożliwiała podgląd 				zmodyfikowanego sygnału przed jego zapisaniem do pliku.

	\section{Opis formatu danych wejściowych/wyjściowych}
Program operuje na plikach Waveform Audio File Format\footnote{\url{https://en.wikipedia.org/wiki/WAV}}.

		\subsection{Dane wejściowe}
W celu wywołania programu, należy podać ścieżkę do pliku z sygnałem wejściowym, ścieżkę do zapisania pliku po przeprowadzeniu filtracji, wybrany rodzaj filtra oraz parametry dla algorytmu:

\begin{itemize}
	\item w przypadku filtru górnoprzepustowego\footnote{\url{https://en.wikipedia.org/wiki/High-pass_filter}}, należy podać dolną granicę częstotliwości;
	\item w przypadku filtru dolnoprzepustowego\footnote{\url{https://en.wikipedia.org/wiki/Low-pass_filter}}, należy podać górną granicę częstotliwości;
	\item w przypadku filtru środkowoprzepustowego\footnote{\url{https://en.wikipedia.org/wiki/Band-pass_filter}} oraz środkowozaporowego\footnote{\url{https://en.wikipedia.org/wiki/Band-stop_filter}}, należy podać obie częstotliwości graniczne.
\end{itemize}

		\subsection{Dane wyjściowe}
Rezultatem wykonania programu jest zapisanie przefiltrowanego pliku falowego we wskazanej przez użytkownika lokalizacji.

Dodatkowo przewiduje się drukowanie postępu wykonania programu na strumień \textit{STDOUT}. W przypadku napotkania błędu, jak niepoprawne kodowanie pliku lub niewłaściwy format parametrów programu, właściwa informacja zostanie wydrukowana na \textit{STDERR}, a program zakończy wykonywanie z niezerowym kodem wyjścia.

	\section{Sieć działań intefejsu}
W podstawowej wersji programu wszystkie parametry podawane będą jako argumenty powłoki tekstowej.

Iterfejs użytkownika dla graficznej wersji programu przedstawiiono na rysunku \ref{fig:interfejs}.

\begin{figure}
	\centering
	\includegraphics[max width=\textwidth,max height=\textheight]{img/interfejs.png}
	\caption{Schemat interfejsu użytkownika dla graficznej wersji programu.}
	\label{fig:interfejs}
\end{figure}

	\section{Sieci działań najważniejszych algorytmów}
Całość programu opiera się na algorytmie filtrowania częstotliwości. Mimo istniania wielu wariantów tego filtra, cały proces można uprościć do postaci przedstawionej na rysunku \ref{fig:algorytm}.

\begin{figure}
	\centering
	\includegraphics[max width=\textwidth,max height=\textheight]{img/algorytm.png}
	\caption{Uproszczony schemat algorytmu filtrowania częstotliwości.}
	\label{fig:algorytm}
\end{figure}


Poszczególne filtry różnić się będą jedynie drugim etapem, czyli sposobem wyboru, które częstotliwości należy wyciąć.

\end{document}